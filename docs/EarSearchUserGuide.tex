\documentclass[11pt]{article}
\usepackage{times, graphicx,float, multicol}
\usepackage[rflt]{floatflt}
\usepackage{hyperref}


\setlength{\textwidth}{6.5in}
\setlength{\textheight}{9.0in}
\setlength{\topmargin}{-.5in}
\setlength{\oddsidemargin}{-.0600in}
\setlength{\evensidemargin}{.0625in}

\newcommand{\secref}[1]{Section~\ref{#1}}

\newcommand{\doublespace}{\baselineskip0.34truein}
\newcommand{\singlespace}{\baselineskip0.16truein}
\newcommand{\midspace}{\baselineskip0.24truein}
\newcommand{\midplusspace}{\baselineskip0.26truein}


\title{Ear Search User Guide\\ {\large Version 1.0}}

\author{Derrick Stolee \\ 
	University of Nebraska-Lincoln\\ 
	\texttt{s-dstolee1@math.unl.edu}
       }
       
\begin{document}

\maketitle
\vspace{-.3in}
\begin{abstract}
	The Ear Search program implements isomorph-free generation
		of 2-connected graphs by ear augmentations.
	This document describes the interfaces used for customized
		searches, as well as describes three example searches:
		unique saturation, edge reconstruction, and 
		extremal graphs with a fixed number of perfect matchings.
\end{abstract}

\section{Introduction}
\label{sec:Introduction}


\section{Data Management}



\section{Pruning}



\section{Canonical Deletion}




\section{Solutions}




\section{Example 0: Enumerating All 2-Connected Graphs}

\section{Example 1: Unique Saturation}


\section{Example 2: Edge Reconstruction}


\section{Example 3: Perfect Matchings}





\bibliographystyle{plain}
\bibliography{bibliography.bib}


\end{document}
