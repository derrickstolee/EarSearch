\documentclass[11pt]{article}
\usepackage{times, graphicx,float, multicol}
\usepackage[rflt]{floatflt}
\usepackage{hyperref}
\usepackage{algorithm,algorithmic}
\usepackage{multirow}

\setlength{\textwidth}{6.5in}
\setlength{\textheight}{9.0in}
\setlength{\topmargin}{-.5in}
\setlength{\oddsidemargin}{-.0600in}
\setlength{\evensidemargin}{.0625in}

\newcommand{\secref}[1]{Section~\ref{#1}}

\newcommand{\doublespace}{\baselineskip0.34truein}
\newcommand{\singlespace}{\baselineskip0.16truein}
\newcommand{\midspace}{\baselineskip0.24truein}
\newcommand{\midplusspace}{\baselineskip0.26truein}


\title{EarSearch User Guide\\ {\large Version 1.0}}

\author{Derrick Stolee \\ 
	University of Nebraska-Lincoln\\ 
	\texttt{s-dstolee1@math.unl.edu}
       }
       
\begin{document}

\maketitle
\vspace{-.3in}
\begin{abstract}
	The Ear Search program implements isomorph-free generation
		of 2-connected graphs by ear augmentations.
	This document describes the interfaces used for customized
		searches, as well as describes three example searches:
		unique saturation, edge reconstruction, and 
		extremal graphs with a fixed number of perfect matchings.
\end{abstract}

\section{Introduction}
\label{sec:Introduction}

The EarSearch library implements the generation algorithm of
	\cite{IsomorphFreeGen} to generate families of 2-connected graphs.
It is based on the TreeSearch library \cite{TreeSearch}.
The class \texttt{EarSearchManager} extends the class \texttt{SearchManager}
	and manages the search tree,
	using ear augmentations to generate children.
It automates the canonical deletion selection in order
	to remove isomorphs.
	

\section{Data Management}


\subsection{Graphs}

Graphs are stored 
	using the \texttt{sparsegraph} structure 
	from the \texttt{nauty} library.

During the course of computation, these graphs are modified using edge and vertex deletions.
To delete the $i$th vertex, set the \texttt{v} array to $-1$ in the $i$th position.
To delete the edge between the $i$ and $j$ vertices,
	set the \texttt{e} array to $-1$ in two places:
	in the list of neighbors for $i$ where $j$ was listed
	and in the list of neighbors for $j$ where $i$ was listed.
To place the vertices or edges back, place the previous values into those places.



\subsection{Augmentations and Labels}

The labels for each augmentation use two 32-bit integers.
The first is the order of the augmented ear.
The second is the index of the pair orbit which is used for
	the endpoints of the ear.

\subsection{\texttt{EarNode}}

Each level of the search tree is stored in a stack, where all data is stored in
	an \texttt{EarNode} object.
All of the members of \texttt{EarNode} are public, in order to easily add data
	structures and flags that are necessary for each application.
All pointers are initialized to $0$ in the constructor and
	are checked to be non-zero before freeing up any memory in the destructor.

The core data necessary for \texttt{EarSearchManager} is stored in 
	the following members:
	
\begin{itemize}
	\item ear\_length -- the length of the augmented ear.
	 

	\item ear -- the byte-array description of the augmented ear.
	 

	\item num\_ears -- the number of ears in the graph.
	 
	\item ear\_list -- the list of ears in the graph (-1 terminated).
	 

	\item graph -- the graph at this node.
	 
	\item max\_verts -- the maximum number of vertices in
	 all supergraphs. Default to max\_n from \texttt{EarSearchManager}.
	 

	\item 
	 reconstructible -- TRUE if detectably reconstructible
	 
	\item numPairOrbits -- the number of pair orbits for this graph.
	 
	 
	\item orbitList -- the list of orbits, in a an array of arrays.
		Each array \texttt{orbitList[i]} contains pair-indices for
			pairs in orbit and is terminated by \texttt{-1}.

	\item canonicalLabels -- the canonical labeling of the graph, 
		stored as an integer array of values for each vertex

	\item solution\_data -- the data of a solution on this node.
	 

	\item violatingPairs -- A set of pair indices which cannot be endpoints of an ear.
	 
\end{itemize}


\section{Pruning}

The interface \texttt{PruningAlgorithm} has an abstract method for
	pruning nodes of the search tree.
The method \texttt{checkPrune} takes two \texttt{EarNode} objects:
	one for the parent and another for the child.
Using this data, the method decides if no solution exists by augmenting
	beyond the child node.
Since the pruning algorithm is called before the canonical deletion algorithm,
	this can also remove nodes which cannot possibly be canonical augmentations.

\section{Canonical Deletion}

The interface \texttt{EarDeletionAlgorithm} has 
	an abstract method for finding a canonical ear deletion.
The method \texttt{getCanonical} takes two \texttt{EarNode} 
	objects for the parent and child
	and returns the array corresponding to the canonical ear.
The \texttt{EarSearchManager} will determine if this
	canonical ear is in orbit with the augmented ear.

\section{Solutions}

The interface \texttt{SolutionChecker} is an abstract class
	which contains methods for finding solutions given
	a search node, storing the solution data, reporting
	on these solutions, and reporting application-specific
	statistics.
	
The method \texttt{isSolution} takes the parent, child, and depth
	and reports if there is a solution at the child node.
It returns a non-null string if and only if there is a solution,
	and that string is a buffer containing the solution data.
This buffer will be deallocated with \texttt{free()} by 
	the \texttt{EarSearchManager}.

The method \texttt{writeStatisticsData()}
	returns a string of statistics (using the TreeSearch format)
	to be reported at the end of a job.



\section{Example 0: 2-Connected Graphs}

To enumerate all 2-connected graphs, the interfaces were 
	implemented to only prune by number of vertices
	and possibly by number of edges.
The search space is defined by three inputs: $N, e_{\min}$, and $e_{\max}$.
These implementations are give by the following classes:

\begin{itemize}
	\item \texttt{EnumeratePruner} will prune a graph if it has more than $N$ vertices
		or more than $e_{\max}$ edges.
		Also, if $e(G) + (N-n(G)+1) > e_{\max}$, it will prune since
			we cannot add the remaining $N-n(G)$ edges without 
			surpassing $e_{\max}$ edges.
		
	\item \texttt{EnumerateDeleter} implements the default deletion algorithm:
			over all ears $e$ in $G$ so that $G-e$ is 2-connected,
			find one of minimum length, then use the canonical labels
			to select the canonical ear.
			
	\item \texttt{EnumerateChecker} detects ``solutions" as any graph
		with exactly $N$ vertices and between $e_{\min}$ and $e_{\max}$ edges.
\end{itemize}


\section{Example 1: Unique Saturation}

The input consists of two numbers $r$ and $N$, 
	and we are searching for uniquely $K_r$-saturated graphs of order $N$.
The unique saturation problem utilizes the deletion algorithm
	in \texttt{EnumerateDeleter}, but adds some data to \texttt{EarNode}
	in order to track the constraints.
The \texttt{SaturationAlgorithm} class implements both the \texttt{PruningAlgorithm}
	and \texttt{SolutionChecker} interfaces.
	
{\bf Note:} The \texttt{SaturationAlgorithm} class is implemented 
	only for $r \in \{4,5,6\}$ in order to use compiler optimizations for
	the nested loop structure.

\subsection{Application-Specific Data}

The following fields were added to \texttt{EarNode} 
	for tracking constraints during the search.
Most information is tracked in \texttt{adj\_matrix\_data},
	which stores information as an adjacency matrix.
The others are boolean flags which mark
	different properties of the current graph.
These flags are set during the \texttt{checkPrune} method,
	and are accessed by the \texttt{isSolution} method.

\begin{itemize}
	\item \texttt{adj\_matrix\_data} -- Data on the (directed) edges.
		For unique saturation, this gives -1 for edges, and 
			for non-edges counts the number of copies of $H$
			given by adding that edge.
		Values are in $\{0,1,2\}$, since when $2$ is 
			listed, then there are too many copies of $H$.

	\item \texttt{any\_adj\_zero} -- A boolean flag: are any of the cells in adj\_matrix\_data zero?
		
	\item \texttt{any\_adj\_two} -- A boolean flag:  are any of the cells in adj\_matrix\_data at least two?
	
	\item \texttt{dom\_vert} -- A boolean flag: is there a dominating vertex?

	\item \texttt{copy\_of\_H} -- A boolean flag: is there a copy of H?

\end{itemize}


\section{Example 2: Edge Reconstruction}

The Edge Reconstruction application takes an integer $N$
	and searches over all 2-connected graphs of order up to $N$
	and up to $1+ \log_2 N!$ edges.
The deletion is built to make graphs with the same deck 
	be siblings.
Then, all siblings which are not detectably edge reconstructible
	are checked to have different edge decks.
	
The following three classes implement the interfaces:

\begin{itemize}
	\item \texttt{ReconstructionPruner} implements the \texttt{PruningAlgorithm}
		interface
		and prunes any graph with more than $N$ vertices or
		more than $1+ \log_2 N!$ edges.
		
	\item \texttt{ReconstructionDeleter} implements the \texttt{EarDeletionAlgorithm}
		interface
		and performs two different deletions:
		
		\begin{enumerate}
			\item If the graph is detectably edge reconstructible, 
				the deletion can be application-ignorant and utilizes
				the standard deletion algorithm from \texttt{EnumerateDeleter}.
				
			\item If the graph is NOT detectably edge reconstructible,
				the canonical ear is selected by using only the edge deck.
				Further, if the deletion is canonical, the graph is stored 
					in the parent \texttt{EarNode} for later comparison of edge decks.
				The \texttt{GraphData} class was implemented specifically for
					storing these children within the parent \texttt{EarNode}.
		\end{enumerate}	
		
	\item \texttt{ReconstructionChecker} implements the \texttt{SolutionChecker}
		interface and compares the current graph's edge deck against 
		all previous siblings.  This is done using three levels of comparison,
			which are implemented in the \texttt{GraphData} class. 
			
\end{itemize}

\subsection{Application-Specific Data}

The \texttt{GraphData} class stores all information for a child graph.
It implements three levels of comparison, which 
	are checked in order within the \texttt{compare} method.

	\begin{enumerate}
		\item \texttt{computeDegSeq} computes and stores the standard degree sequence for the
			current graph.
			
		\item \texttt{computeInvariant} calculates and stores  a more complicated function 
			based on the degree sequence and the degrees of the neighborhood for 
			each vertex.
			
		\item \texttt{computeCanonStrings} computes canonical strings for every edge-deleted
			subgraph and sorts the list. These are then compared, card-for-card.
	\end{enumerate}
	
In order to store these \texttt{GraphData} objects, 
	the following members were added to the \texttt{EarNode}
	class:
	
\begin{itemize}
	
	\item child\_data -- the GraphData objects for immediate children, used for pairwise comparison.
	
	\item num\_child\_data -- the number of GraphData objects currently filling the data.
	\item size\_child\_data -- the number of pointers currently allocated.
\end{itemize}


\section{Example 3: $p$-Extremal Graphs}

This problem is investigated in \cite{pExtremal}
	and is the most involved of all applications.
See \cite{DudekSchmitt} and \cite{HSWY} for background on this problem.
The input is given as $P_{\min}$, $P_{\max}$, $C$, and $N$.
The search is for elementary graphs with $p$ perfect matchings
	(for $P_{\min} \leq p \leq P_{\max}$)
	with excess at least $C$ and at most $N$ vertices.
The search actually runs over 1-extendable and almost 1-extendable graphs,
	which are the graphs reachable by the ear augmentations.
A second stage adds forbidden edges to maximize excess
	without increasing the number of perfect matchings.
	
The following classes implement the EarSearch interfaces:

\begin{itemize}
	\item \texttt{MatchingPruner} implements the 
		\texttt{PruningAlgorithm} interface.
		Graphs are pruned for three reasons:
		
		\begin{enumerate}
			\item There are an odd number of vertices.
				By the Lov\'asz Two Ear Theorem, we know 
					that every ear augmentation 
					has an even number of internal vertices.
					
			\item There are more than $P_{\max}$ perfect matchings.
			
			\item The parent graph was not 1-extendable, and neither is the current graph.
				By the Lov\'asz Two Ear Theorem, we can always go from 1-extendable
				to 1-extendable using at most two ear augmentations.
				
			\item Let $c$ be the maximum excess of an elementary supergraph
				of the current graph, which is of order $n$,
				and let $p$ be the current number of perfect matchings.
				If $c+2(P_{\max}-p)-\frac{1}{4}(n'-n)(n-2) < C$
				for all $n \leq n' \leq N$, then prune.
				Otherwise, maximize the $n'$ so that 
					the inequality 
					$c+2(P_{\max}-p)-\frac{1}{4}(n'-n)(n-2) \geq C$ holds.
				That value of $n'$ is then used to bound the 
					length of future ear augmentations,
					since no graph reachable from the current graph
					can have excess at least $C$ and more than $n'$ vertices.
		\end{enumerate}
		
		In addition to pruning, the pruning algorithm
			also performs the on-line algorithm
			for updating the list of barriers by using the current ear 
			augmentation.
		
	\item \texttt{MatchingChecker} implements the
		\texttt{SolutionChecker} interface.
		Given a 1-extendable graph with between $P_{\min}$ and $P_{\max}$
			perfect matchings, 
			forbidden edges are added in all possible ways
			and the elementary supergraphs with excess at least $C$
			are printed to output.
		If any are found, the \texttt{isSolution} method returns with success.
		The algorithm for enumerating all 
			elementary supergraphs is implemented in the
			\texttt{BarrierSearch.cpp} file.
		
	\item \texttt{MatchingDeleter} implements the
		\texttt{EarDeletionAlgorithm} interface.
		The following sequence of choices 
			describe the method for selecting 
			a canonical ear to delete from a graph $H$:
		
		\begin{enumerate}
			\item If $H$ is almost 1-extendable, 
				we need to delete an ear $e'$ so that $H - e'$ is 1-extendable.
				By the definition of almost 1-extendable, there is a unique such choice.
				
			\item If $H$ is 1-extendable, 
				check if there exists an ear $e'$ so that $H - e'$ is 1-extendable.
				If one exists, select one of minimum length and break ties
					using the canonical labels of the endpoints.
					
			\item If $H$ is 1-extendable and 
					no single ear $e'$ makes $H - e'$ 1-extendable, then
					find an ear $e$ so that there is a disjoint ear $f$ 
					with $H-e$ is almost 1-extendable
					and
					$H - e - f$ is 1-extendable.
					Out of these choices for $e$, choose one of 
						minimum length and break ties
					using the canonical labels of the endpoints.
		\end{enumerate}
\end{itemize}

\subsection{Application-Specific Data}

The following members were added to \texttt{EarNode} 
	to help the perfect matchings application.

\begin{itemize}

	\item \texttt{extendable} -- A boolean flag: is the graph 1-extendable?

	\item  \texttt{numMatchings} -- The number of perfect matchings for this graph.

	\item  \texttt{barriers} -- The list of barriers of the graph, given as an array of \texttt{Set}
		pointers.  This barrier list is updated at each level by an
		 on-line algorithm. 

	\item  \texttt{num\_barriers} -- the number of barriers in the graph.

\end{itemize}

\subsection{Perfect Matching Algorithms}

There are a few algorithms that are implemented in order 
	to solve certain sub-problems, such as counting perfect matchings
	or enumerating independent sets.
These are computationally complex problems, but 
	the implementations are very fast for these small instances.
The algorithms are mostly un-optimized and rely on simple instructions and 
	low overhead in order to be run many many times during the course
	of the search.
	
\begin{itemize}
	\item \texttt{countPM}($G,P$) counts the number of perfect matchings in a graph $G$,
		with an upper bound of $P$.
		It operates recursively, selecting an edge $e$ in $G$ and 
			attempts to extend the current matching using $e$ and not using $e$.
		When a perfect matching is found, the counter increases.
		There are two shortcutting strategies:
		
		\begin{enumerate}
			\item If there is ever a vertex with no available edges, 
				the recursion is halted with a count of zero perfect matchings,
				since the current matching does not extend to a perfect matching.
				
			\item If the current count of perfect matchings ever surpasses $P$,
				then the current value is returned.
				During the search, we only care about graphs with at most $P_{\max}$
					perfect matchings, so graphs with many more
					will only be pruned.
		\end{enumerate} 
		
	\item \texttt{isExtendable}($G$) tests if the given graph is 1-extendable.
		This is done by storing an array of boolean flags for each edge, marking 
			each as they are found to be in perfect matchings.
		This algorithm is explicitly used in the deletion algorithm.
		During the pruning algorithm, where a specific augmentation is given, 
			we can detect 1-extendability by 
			asking if there is a perfect matching using the 
			proper alternating path within the augmented ear.
	
	\item \texttt{enumerateAllBarrierExtensions}($G, {\mathcal B}, C$) 
		and \texttt{searchAllBarrierExtensions}($G, {\mathcal B}$)
		are two methods which take a 1-extendable graph $G$ with barrier list $\mathcal B$
		and attempts to add forbidden edges to $G$ to attain the maximum excess.
		The difference is that \texttt{enumerateAllBarrierExtensions} will
			output any graphs with excess at least $C$, 
			while \texttt{searchAllBarrierExtensions} will simply return the largest excess.
		The algorithm essentially enumerates all independent sets within
			the barrier conflict graph ${\mathcal B}$, where
			conflicts are computed on the fly.
		The enumeration is recursive, simply testing if
			the next available barrier should be added to the current independent set.
		As each set is added, it tests which barriers 
			with larger index are in conflict with this graph.
		These barriers are then not considered in deeper recursive calls.
		Due to the low overhead for each independent set, this
			simple algorithm runs fast enough for the search to be feasible.
\end{itemize}



\begin{thebibliography}{9}
	\bibitem{DudekSchmitt} A. Dudek and J. Schmidt.
		On extremal graphs with a constant number of 1-factors.
		submitted. 2010.

	\bibitem{HRnauty} S. G. Hartke and A. J. Radcliffe. 
		Mckay's canonical graph labeling algorithm. 
		In {\it Communicating Mathematics}, 
		volume 479 of {\it Contemporary Mathematics}, 99Ð111. 2009.
		
	\bibitem{HSWY} S. G. Hartke, D. Stolee, D. B. West, and M. Yancey.
		On extremal graphs with a fixed number of perfect matchings.
		in preparation. 2011.

	\bibitem{McKayReconstruct} B. D. McKay. Small graphs are reconstructible. 
		{\it Australas. J. Combin.}, 15:123Ð 126, 1997.

	\bibitem{McKayGenerate} B. D. McKay, 
		Isomorph-free exhaustive generation
		{\it J. Algorithms}, 26(6):306-324. 1998.
		
	\bibitem{McKayNauty} B. D. McKay,
		nauty user's guide (version 2.4)
		Dept. Computer Science, Astral. Nat. Univ., 2006.
	
	\bibitem{IsomorphFreeGen} D. Stolee, 
		Isomorph-free generation of 2-connected graphs with applications,
		in preparation, 2011.
		
	\bibitem{pExtremal} D. Stolee,
		Generating $p$-extremal graphs,
		in preparation, 2011.
		
	\bibitem{TreeSearch} D. Stolee,
		TreeSearch user guide,
		available at 
		\href{http://www.github.com/derrickstolee/TreeSearch/}{http://www.github.com/derrickstolee/TreeSearch/}
		2011.
\end{thebibliography}

\end{document}
